\documentclass[11pt]{article}
\usepackage[english]{babel}
\usepackage[T1]{fontenc}
\usepackage[utf8]{inputenc}
\usepackage{amsmath,amssymb,amsthm,geometry}
\usepackage{graphicx}
\usepackage{tikz}
\usetikzlibrary{calc}
\usepackage{hyperref}
\geometry{margin=2.5cm}

\title{The Dodecagram Polygonal Theorem}
\author{Jose Eugenio A. G.}
\date{02/19/2026}

\newtheorem{theorem}{Theorem}
\newtheorem{lemma}{Lemma}
\newtheorem{definition}{Definition}
\newtheorem{corollary}{Corollary}

\begin{document}
\maketitle

\begin{abstract}
We study regular star polygons $\{n/k\}$ satisfying the Diophantine relation
$k = n/2 - 1$. We prove that the only such polygons whose non-convex
internal stellation rings are all composite are $\{8/3\}$ and $\{12/5\}$,
and that $\{12/5\}$ is maximal in $n$. This identifies the regular
dodecagram as a sharp arithmetic threshold in the structure of regular
stellations.
\end{abstract}

\section{Introduction}

Regular star polygons constitute a classical topic in Euclidean geometry and are denoted by the Schläfli symbol $\{n/k\}$. Their stellations generate internal rings whose components may be simple polygons, star polygons, or unions of multiple regular polygons.

We introduce a critical arithmetic relation between $n$ and $k$ and show that the dodecagram $\{12/5\}$ occupies a unique maximal structural position within this family.

\section{Preliminaries}

\begin{definition}
A \emph{regular star polygon} $\{n/k\}$ consists of $n$ equally spaced vertices on a circle, with edges joining each vertex to the $k$-th subsequent vertex, where $\gcd(n,k)=1$ and $1<k<n/2$.
\end{definition}

\begin{definition}
An \emph{internal ring} of $\{n/k\}$ is the orbit, under the action of the dihedral group $D_n$, of the distinct intersection points of pairs of non-adjacent, non-parallel edges that lie on a common concentric circle.
\end{definition}

\begin{definition}
A ring with step size $s$ is \emph{composite} if $\gcd(n,s)>1$, in which case it consists of $\gcd(n,s)$ congruent regular polygons. A ring is \emph{simple} if $\gcd(n,s)=1$.
\end{definition}

\section{Critical Diophantine Relation}

We impose the Diophantine condition
\[
k = \frac{n}{2} - 1, \qquad n = 2(k+1).
\]

\section{Explicit Ring Geometry}

\begin{lemma}[Explicit ring radius]
For the regular star polygon $\{n/k\}$ inscribed in the unit circle,
the radius of the $m$-th internal ring is

\[
r_m=
\frac{
\sin\left(\frac{\pi m}{n}\right)
\sin\left(\frac{\pi (k-m)}{n}\right)
}{
\sin^2\left(\frac{\pi k}{n}\right)
},
\quad 1\le m\le k-1.
\]
\end{lemma}

\begin{lemma}[Monotonicity of ring radii]
For the regular star polygon $\{n/k\}$ with $1<k<n/2$,
the internal ring radii satisfy
\[
r_1<r_2<\cdots<r_{k-1}.
\]
\end{lemma}

\begin{proof}
Since the denominator of
\[
r_m=\frac{\sin\left(\frac{\pi m}{n}\right)
\sin\left(\frac{\pi(k-m)}{n}\right)}
{\sin^2\left(\frac{\pi k}{n}\right)}
\]
is constant in $m$, it suffices to prove the monotonicity of

\[
f(m)=\sin\left(\frac{\pi m}{n}\right)
\sin\left(\frac{\pi(k-m)}{n}\right).
\]

Using $\sin a\sin b=\tfrac12[\cos(a-b)-\cos(a+b)]$, we obtain

\[
f(m)=\frac12\left[
\cos\left(\frac{\pi(2m-k)}{n}\right)
-\cos\left(\frac{\pi k}{n}\right)
\right].
\]

The second term is constant in $m$, so monotonicity is determined by
$g(m)=\cos\left(\frac{\pi(2m-k)}{n}\right)$.

Since $1<k<n/2$, we have
\[
\left|\frac{\pi(2m-k)}{n}\right|<\frac{\pi}{2}
\quad\text{for }1\le m\le k-1.
\]
On this interval the cosine function is strictly decreasing in its argument,
while the argument increases linearly in $m$.
Hence $g(m)$, and therefore $r_m$, is strictly increasing in $m$.
\end{proof}

\section{Main Result}

\begin{theorem}[Dodecagram Polygonal Theorem]
Let $\{n/k\}$ be a regular star polygon satisfying
\[
k = \frac{n}{2} - 1.
\]

Then:

\begin{enumerate}
\item The polygon has exactly $k-1$ internal rings with step sizes $s=k-1,\dots,1$.

\item The only integer solutions $(n,k)$ for which all rings with $s\ge 2$ are composite are
\[
(8,3)\quad\text{and}\quad(12,5).
\]

\item Among these, $\{12/5\}$ is maximal in $n$.
\end{enumerate}
\end{theorem}

\section{Number-Theoretic Lemmas}

\begin{lemma}
If $k=n/2-1$ and $\gcd(n,k)=1$, then $k$ is odd.
\end{lemma}

\begin{proof}
Since $n=2(k+1)$,
\[
\gcd(n,k)=\gcd(2(k+1),k)=\gcd(2,k).
\]
Thus $\gcd(n,k)=1$ iff $k$ is odd.
\end{proof}

\begin{lemma}
Let $k$ be odd and $n=2(k+1)$. Then $\gcd(n,s)>1$ for all $s\in\{2,\dots,k-1\}$ if and only if $(n,k)=(8,3)$ or $(n,k)=(12,5)$.
\end{lemma}

\begin{proof}
The condition requires that every integer in $\{2,\dots,k-1\}$ shares a prime factor with $n=2(k+1)$.

The cases $(8,3)$ and $(12,5)$ satisfy this directly.

For odd $k\ge 7$, let $p_0$ be the smallest prime not dividing $n$. Standard prime bounds imply $p_0<k$, hence $s=p_0$ gives $\gcd(n,s)=1$, producing a simple ring.

Therefore, for every odd $k \ge 7$, there exists
$s \in \{2,\dots,k-1\}$ such that $\gcd(n,s)=1$,
so at least one internal ring is simple.
\end{proof}

\section{Conclusion}

We have proven that $\{12/5\}$ is the maximal regular star polygon satisfying $k=n/2-1$ whose non-convex internal rings are all composite. The result follows from elementary number theory and symmetry arguments and reveals a sharp arithmetic threshold in stellation structure.

\end{document}