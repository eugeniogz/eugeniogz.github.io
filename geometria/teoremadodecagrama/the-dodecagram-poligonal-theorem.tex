\documentclass[11pt]{article}

\usepackage[english]{babel}
\usepackage[T1]{fontenc}
\usepackage[utf8]{inputenc}
\usepackage{amsmath,amssymb,amsthm}
\usepackage{geometry}
\usepackage{hyperref}
\usepackage{tikz}
\usetikzlibrary{calc}
\geometry{margin=2.5cm}

\title{The Dodecagram Polygonal Theorem}

\author{
Jose Eugenio A. G.\\
\small \texttt{(jeugenio@ufmg.br)}
}

\date{\today}

\newtheorem{theorem}{Theorem}
\newtheorem{lemma}{Lemma}
\newtheorem{definition}{Definition}
\newtheorem{corollary}{Corollary}

\begin{document}
\maketitle

\begin{abstract}
We study regular star polygons $\{n/k\}$ satisfying the Diophantine relation
$k = n/2 - 1$. We prove that the only such polygons whose non-convex
internal stellation rings are all composite are $\{8/3\}$ and $\{12/5\}$,
and that $\{12/5\}$ is maximal in $n$. This identifies the regular
dodecagram as a sharp arithmetic threshold in the structure of regular
stellations.
\end{abstract}

\noindent\textbf{MSC (2020):} 51M20, 11A05

\noindent\textbf{Keywords:} star polygons, stellation, dodecagram, discrete geometry

% =========================================================

\section{Introduction}

Regular star polygons constitute a classical topic in Euclidean geometry and are denoted by the Schläfli symbol $\{n/k\}$. Their stellations generate internal rings whose components may be simple polygons, star polygons, or unions of multiple regular polygons. Classical treatments may be found in \cite{coxeter1973,coxeter1938,grunbaum2003}.

We introduce a critical arithmetic relation between $n$ and $k$ and show that the dodecagram $\{12/5\}$ occupies a unique maximal structural position within this family.

% =========================================================

\section{Preliminaries}

\begin{definition}
A \emph{regular star polygon} $\{n/k\}$ consists of $n$ equally spaced vertices on a circle, with edges joining each vertex to the $k$-th subsequent vertex, where $\gcd(n,k)=1$ and $1<k<n/2$.
\end{definition}

\begin{definition}
An \emph{internal ring} of $\{n/k\}$ is the orbit, under the action of the dihedral group $D_n$, of the distinct intersection points of pairs of non-adjacent, non-parallel edges that lie on a common concentric circle.
\end{definition}

\begin{definition}
A ring with step size $s$ is \emph{composite} if $\gcd(n,s)>1$, in which case it consists of $\gcd(n,s)$ congruent regular polygons. A ring is \emph{simple} if $\gcd(n,s)=1$.
\end{definition}

% =========================================================

\section{Critical Relation}

We impose the Diophantine condition
\[
k = \frac{n}{2} - 1, \qquad n = 2(k+1).
\]

% =========================================================

\section{Ring Geometry}

\begin{lemma}[Explicit ring radius]
For the regular star polygon $\{n/k\}$ inscribed in the unit circle,
the radius of the $m$-th internal ring is

\[
r_m=
\frac{
\sin\left(\frac{\pi m}{n}\right)
\sin\left(\frac{\pi (k-m)}{n}\right)
}{
\sin^2\left(\frac{\pi k}{n}\right)
},
\quad 1\le m\le k-1.
\]
\end{lemma}

\begin{lemma}[Monotonicity]
For $1<k<n/2$, the radii satisfy
\[
r_1<r_2<\cdots<r_{k-1}.
\]
\end{lemma}

\begin{proof}
Since the denominator of
\[
r_m=\frac{\sin\left(\frac{\pi m}{n}\right)
\sin\left(\frac{\pi(k-m)}{n}\right)}
{\sin^2\left(\frac{\pi k}{n}\right)}
\]
is constant in $m$, it suffices to prove the monotonicity of

\[
f(m)=\sin\left(\frac{\pi m}{n}\right)
\sin\left(\frac{\pi(k-m)}{n}\right).
\]

Using $\sin a\sin b=\tfrac12[\cos(a-b)-\cos(a+b)]$, we obtain

\[
f(m)=\frac12\left[
\cos\left(\frac{\pi(2m-k)}{n}\right)
-\cos\left(\frac{\pi k}{n}\right)
\right].
\]

The second term is constant in $m$, so monotonicity is determined by
$g(m)=\cos\left(\frac{\pi(2m-k)}{n}\right)$.

Since $1<k<n/2$, we have
\[
\left|\frac{\pi(2m-k)}{n}\right|<\frac{\pi}{2}
\quad\text{for }1\le m\le k-1.
\]
Hence $r_m$ is strictly increasing in $m$.
\end{proof}

% =========================================================

\section{Number-Theoretic Lemmas}

\begin{lemma}[Primorial growth]
Let $p\#$ denote the product of all primes $\le p$.
For $p\ge5$ we have
\[
p\# > 2p.
\]
\end{lemma}

\begin{proof}
Direct verification gives $5\#=30>10$.
For larger $p$, the product gains an additional prime factor at least $p$,
so the inequality continues to hold; see \cite{hardywright}.
\end{proof}

% =========================================================

\section{Main Theorem}

\begin{theorem}[Dodecagram Polygonal Theorem]
Let $\{n/k\}$ satisfy $k=n/2-1$.

Then the only solutions for which all rings with $s\ge2$ are composite are
$(8,3)$ and $(12,5)$, and $\{12/5\}$ is maximal in $n$.
\end{theorem}

\begin{proof}
Assume $n=2(k+1)$ with $k$ odd.

The internal rings correspond to steps
\[
s\in\{1,2,\dots,k-1\}.
\]

A ring is composite iff $\gcd(n,s)>1$.
We require this for all $s\in\{2,\dots,k-1\}$.

\medskip
\textbf{Small cases.}

For $k=3$, $n=8$ works.

For $k=5$, $n=12$ works.

\medskip
\textbf{General case $k\ge7$.}

Let $q$ be the largest prime dividing $n$ among $3,5,\dots$.
If all primes up to $q$ divide $n$, then
\[
2\cdot3\cdot5\cdots q \mid (k+1),
\]
so $k+1\ge q\#/2$.

By Lemma above, for $q\ge5$ we have $q\#/2>q$,
hence $k\ge q+1$.

Let $q'$ be the next prime after $q$.
Then $q'\le k-1$ and $q'\nmid n$, so
\[
\gcd(n,q')=1,
\]
producing a simple ring.

Thus no $k\ge7$ satisfies the condition.
The only solutions are $(8,3)$ and $(12,5)$, and the latter is maximal.
\end{proof}

% =========================================================

\section{Conclusion}

We identified $\{12/5\}$ as the maximal member of a family of regular star polygons whose internal non-convex rings remain fully composite.

\bigskip
\noindent\textbf{Acknowledgments.}
The author thanks discussions in discrete geometry that motivated this work.

% =========================================================

\begin{thebibliography}{9}

\bibitem{coxeter1973}
H. S. M. Coxeter,
\emph{Regular Polytopes},
3rd ed., Dover, 1973.

\bibitem{coxeter1938}
H. S. M. Coxeter, P. Du Val, H. T. Flather, and J. F. Petrie,
\emph{The Fifty-Nine Icosahedra},
University of Toronto Press, 1938.

\bibitem{grunbaum2003}
Branko Grünbaum,
\emph{Convex Polytopes},
2nd ed., Springer, 2003.

\bibitem{hardywright}
G. H. Hardy and E. M. Wright,
\emph{An Introduction to the Theory of Numbers},
Oxford University Press, 1979.

\end{thebibliography}

\end{document}