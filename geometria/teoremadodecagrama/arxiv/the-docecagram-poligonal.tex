\documentclass[11pt]{article}

% --- arXiv-safe packages ---
\usepackage[english]{babel}
\usepackage[T1]{fontenc}
\usepackage[utf8]{inputenc}
\usepackage{amsmath,amssymb,amsthm}
\usepackage{geometry}
\usepackage{hyperref}
\usepackage{tikz}
\usetikzlibrary{calc}
\geometry{margin=2.5cm}

% --- metadata ---
\title{The Dodecagram Polygonal Theorem}

\author{
Jose Eugenio A. G.\\
\small Independent Researcher\\
\small \texttt{(optional email)}
}

\date{\today}

% --- theorem environments ---
\newtheorem{theorem}{Theorem}
\newtheorem{lemma}{Lemma}
\newtheorem{definition}{Definition}
\newtheorem{corollary}{Corollary}

\begin{document}
\maketitle

\begin{abstract}
We study regular star polygons $\{n/k\}$ satisfying the Diophantine relation
$k = n/2 - 1$. We prove that the only such polygons whose non-convex
internal stellation rings are all composite are $\{8/3\}$ and $\{12/5\}$,
and that $\{12/5\}$ is maximal in $n$. This identifies the regular
dodecagram as a sharp arithmetic threshold in the structure of regular
stellations.
\end{abstract}

\noindent\textbf{MSC (2020):} 51M20, 11A05

\noindent\textbf{Keywords:} star polygons, stellation, dodecagram, discrete geometry

% =========================================================

\section{Introduction}

Regular star polygons constitute a classical topic in Euclidean geometry and are denoted by the Schläfli symbol $\{n/k\}$. Their stellations generate internal rings whose components may be simple polygons, star polygons, or unions of multiple regular polygons.

We introduce a critical arithmetic relation between $n$ and $k$ and show that the dodecagram $\{12/5\}$ occupies a unique maximal structural position within this family.

% =========================================================

\section{Preliminaries}

\begin{definition}
A \emph{regular star polygon} $\{n/k\}$ consists of $n$ equally spaced vertices on a circle, with edges joining each vertex to the $k$-th subsequent vertex, where $\gcd(n,k)=1$ and $1<k<n/2$.
\end{definition}

\begin{definition}
An \emph{internal ring} of $\{n/k\}$ is the orbit, under the action of the dihedral group $D_n$, of the distinct intersection points of pairs of non-adjacent, non-parallel edges that lie on a common concentric circle.
\end{definition}

\begin{definition}
A ring with step size $s$ is \emph{composite} if $\gcd(n,s)>1$, in which case it consists of $\gcd(n,s)$ congruent regular polygons. A ring is \emph{simple} if $\gcd(n,s)=1$.
\end{definition}

% =========================================================

\section{Critical Relation}

We impose the Diophantine condition
\[
k = \frac{n}{2} - 1, \qquad n = 2(k+1).
\]

% =========================================================

\section{Ring Geometry}

\begin{lemma}[Explicit ring radius]
For the regular star polygon $\{n/k\}$ inscribed in the unit circle,
the radius of the $m$-th internal ring is

\[
r_m=
\frac{
\sin\left(\frac{\pi m}{n}\right)
\sin\left(\frac{\pi (k-m)}{n}\right)
}{
\sin^2\left(\frac{\pi k}{n}\right)
},
\quad 1\le m\le k-1.
\]
\end{lemma}

\begin{lemma}[Monotonicity]
For $1<k<n/2$, the radii satisfy
\[
r_1<r_2<\cdots<r_{k-1}.
\]
\end{lemma}

\begin{proof}
(Same proof you already have — keep it.)
\end{proof}

% =========================================================

\section{Main Theorem}

\begin{theorem}[Dodecagram Polygonal Theorem]
Let $\{n/k\}$ satisfy $k=n/2-1$.

Then the only solutions for which all rings with $s\ge2$ are composite are
$(8,3)$ and $(12,5)$, and $\{12/5\}$ is maximal in $n$.
\end{theorem}

% =========================================================

\section{Conclusion}

We identified $\{12/5\}$ as the maximal member of a family of regular star polygons whose internal non-convex rings remain fully composite.

\bigskip
\noindent\textbf{Acknowledgments.}
The author thanks discussions in discrete geometry that motivated this work.

\end{document}